\documentclass[10pt,twocolumn,letterpaper]{article}

\usepackage{cvpr}
\usepackage{times}
\usepackage{epsfig}
\usepackage{graphicx}
\usepackage{amsmath}
\usepackage{amssymb}

% Include other packages here, before hyperref.

% If you comment hyperref and then uncomment it, you should delete
% egpaper.aux before re-running latex.  (Or just hit 'q' on the first latex
% run, let it finish, and you should be clear).
\usepackage[breaklinks=true,bookmarks=false]{hyperref}

\cvprfinalcopy % *** Uncomment this line for the final submission

\def\cvprPaperID{****} % *** Enter the CVPR Paper ID here
\def\httilde{\mbox{\tt\raisebox{-.5ex}{\symbol{126}}}}

% Pages are numbered in submission mode, and unnumbered in camera-ready
%\ifcvprfinal\pagestyle{empty}\fi
\setcounter{page}{4321}
\begin{document}

%%%%%%%%% TITLE
\title{\LaTeX\ Author Guidelines for CVPR Proceedings}
\author{Yujie Liu\qquad Yumeng Liu\qquad Zetian Xiao\\University of Rochester}


\maketitle
%\thispagestyle{empty}

%%%%%%%%% BODY TEXT
\section{Team Information}

%-------------------------------------------------------------------------
\subsection{Team name}


\subsection{Team members}
\begin{itemize}
\item Yujie Liu (yliu134@u.rochester.edu)
\item Yumeng Liu (yliu114@u.rochester.edu)
\item Zetian Xiao (*@u.rochester.edu)
\end{itemize}

%------------------------------------------------------------------------
\section{Problem and our approach}


%------------------------------------------------------------------------
\section{Related work}

You must include your signed IEEE copyright release form when you submit
your finished paper. We MUST have this form before your paper can be
published in the proceedings.

\section{Potential dataset}
PASCAL VOC (Visual Object Classes) 2012: a benchmark in visual object category recognition and  detection,  providing  the  vision  and  machine  learning
communities with a standard dataset of images and annotation, and standard evaluation procedures. 
\section{Timeline}
\begin{center}
	\begin{tabularx}{8cm}{|X|X|}
	\hline
	Date & Task \\
	\hline
	April 7th & Understand all related works and implement the original FCN \\
	\hline
	April 14th & Brainstorm and try new improvements for the algorithm based on the running results\\
	\hline
	April 21st & Choose one specific improvement with best outcomes and finalize all the details  \\
	\hline
	April 25th & Finish preparation for presentation\\
	\hline
	May 15th & Finalize the paper\\
	\hline
	\end{tabularx}
\end{center}


{\small
\bibliographystyle{ieee}
\bibliography{egbib}
}


xample~\cite{Authors14}.  Where appropriate, include the name(s) of
editors of referenced books.
\url{https://scholar.google.com/scholar?hl=en&as_sdt=0,33&q=The+P+ASCAL+Visual+Object+Classes+(VOC)+Challenge&btnG}




Introduction

The main goal of this challenge is to recognize objects from a number of visual object classes in realistic scenes (i.e. not pre-segmented objects). It is fundamentally a supervised learning learning problem in that a training set of labelled images is provided. The twenty object classes that have been selected are:

    Person: person
    Animal: bird, cat, cow, dog, horse, sheep
    Vehicle: aeroplane, bicycle, boat, bus, car, motorbike, train
    Indoor: bottle, chair, dining table, potted plant, sofa, tv/monitor

There are three main object recognition competitions: classification, detection, and segmentation, a competition on action classification, and a competition on large scale recognition run by ImageNet. In addition there is a "taster" competition on person layout.
Classification/Detection Competitions

    Classification: For each of the twenty classes, predicting presence/absence of an example of that class in the test image.
    Detection: Predicting the bounding box and label of each object from the twenty target classes in the test image. 
    
Introduction

The main goal of this challenge is to recognize objects from a number of visual object classes in realistic scenes (i.e. not pre-segmented objects). It is fundamentally a supervised learning learning problem in that a training set of labelled images is provided. The twenty object classes that have been selected are:

    Person: person
    Animal: bird, cat, cow, dog, horse, sheep
    Vehicle: aeroplane, bicycle, boat, bus, car, motorbike, train
    Indoor: bottle, chair, dining table, potted plant, sofa, tv/monitor

There are three main object recognition competitions: classification, detection, and segmentation, a competition on action classification, and a competition on large scale recognition run by ImageNet. In addition there is a "taster" competition on person layout.
Classification/Detection Competitions

    Classification: For each of the twenty classes, predicting presence/absence of an example of that class in the test image.
    Detection: Predicting the bounding box and label of each object from the twenty target classes in the test image. 
    
Introduction

The main goal of this challenge is to recognize objects from a number of visual object classes in realistic scenes (i.e. not pre-segmented objects). It is fundamentally a supervised learning learning problem in that a training set of labelled images is provided. The twenty object classes that have been selected are:

    Person: person
    Animal: bird, cat, cow, dog, horse, sheep
    Vehicle: aeroplane, bicycle, boat, bus, car, motorbike, train
    Indoor: bottle, chair, dining table, potted plant, sofa, tv/monitor

There are three main object recognition competitions: classification, detection, and segmentation, a competition on action classification, and a competition on large scale recognition run by ImageNet. In addition there is a "taster" competition on person layout.
Classification/Detection Competitions

    Classification: For each of the twenty classes, predicting presence/absence of an example of that class in the test image.
    Detection: Predicting the bounding box and label of each object from the twenty target classes in the test image. 
    
\end{document}
